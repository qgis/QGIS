% html: Beginning of file: `index.html'
    
\section{Using the Delmited Text Plugin}
\label{f0}    

The Delimited Text plugin allows you to load a delimited text file  as a layer in QGIS. 
    
\subsection{Requirements}
To view a delimited text file as layer, the text file must contain:
\begin{enumerate}      
\item A delimited header row of field names. This must be the first line in the text file     \item The header row must contain an X and Y field. These fields can have any name.
\item The x and y coordinates must be specified as a number. The coordinate system is not important
\end{enumerate}
An example of a valid text file might look like this:
\begin{verbatim} 
name|latdec|longdec|cell|
196 mile creek|61.89806|-150.0775|tyonek d-1 ne|
197 1/2 mile creek|61.89472|-150.09972|tyonek d-1 ne|
a b mountain|59.52889|-135.28333|skagway c-1 sw|
apw dam number 2|60.53|-145.75167|cordova c-5 sw|
apw reservoir|60.53167|-145.75333|cordova c-5 sw|
apw reservoir|60.53|-145.75167|cordova c-5 sw|
aaron creek|56.37861|-131.96556|bradfield canal b-6|
aaron island|58.43778|-134.81944|juneau b-3 ne|
aats bay|55.905|-134.24639|craig d-7|
\end{verbatim}


Some items of note about the text file are:

\begin{enumerate}        
\item  The example text file uses \mbox{$|$} as delimter. Any character can be used to         delimit the fields.
\item The first row is the header row. It contains the fields name, latdec, longdec, and cell
\item No quotes ({\tt{}"{}}) are used to delimit text fields
\item The x coordinates are contained in the {\em longdec} field
\item The y coordinates are contained in the {\em latdec} field
\end{enumerate}


\subsection{Using the Plugin}
To use the plugin you must have QGIS running and use the Plugin Manager to load the plugin:

Start QGIS, then Open the Plugin Manager by choosing the {\em Tools\mbox{$|$}Plugin Manager} menu. The Plugin Manager displays a list of available plugins. Plugins that are already loaded have a checkmark to the left of their name. Click on the checkbox to the left of the {\em Add Delimited Text Layer} plugin and click Ok to load it as shown in Figure \ref{fig:plugin_manager}.

\begin{figure}[h]
   \begin{center}
   \caption{Plugin Manager Dialog}
   \label{fig:plugin_manager}
   \smallskip
   \includegraphics[scale=0.6]{qgis_user_guide_images/plugins_delimited_text/plugin_manager}
   \end{center}  
\end{figure}


A new toolbar icon is now present:
\includegraphics[scale=0.8]{qgis_user_guide_images/plugins_delimited_text/toolbar_icon}
Click on the icon to open the Delimited Text dialog as shown in Figure \ref{fig:delim_text_plugin_dialog}.

\begin{figure}[h]
   \begin{center}
   \caption{Delimited Text Dialog}\label{fig:delim_text_plugin_dialog}\smallskip
   \includegraphics[scale=0.6]{qgis_user_guide_images/plugins_delimited_text/dialog}            
   \end{center}  
\end{figure}

  
First select the file to import by clicking on the ellipsis button: 
\includegraphics[scale=0.5]{qgis_user_guide_images/plugins_delimited_text/ellipsis}
Select the desired text file from the file dialog
Once the file is selected, the plugin attempts to parse the file using the last
used delimiter, in this case \mbox{$|$} (Figure
\ref{fig:delim_text_file_selected}).
\begin{figure}[h]
   \begin{center}
   \caption{File Selected}\label{fig:delim_text_file_selected}\smallskip
   \includegraphics[scale=0.6]{qgis_user_guide_images/plugins_delimited_text/file_selected}   
   \end{center}  
\end{figure}

In this case the delimiter \mbox{$|$} is not correct for the file. The file is actually tab delimited. Note that the X and Y field drop down boxes do not contain valid field names.
To properly parse the file, change the delimiter to tab using
\mbox{$\backslash$}t (this is a regular expression for the tab character). After
changing the delimiter, click {\em Parse}.
The drop down boxes now contain the fields properly parsed as shown in Figure
\ref{fig:delim_text_file_selected2}.

Choose the X and Y fields from the drop down boxes and enter a Layer name as
shown in Figure \ref{fig:delim_text_file_selected3}. To add the layer to the
map, click {\em Add Layer}. The delimited text file now behaves as any other map
layer in QGIS (Figure \ref{fig:layer_added}).
\begin{figure}[h]
   \begin{center}
   \caption{Fields Parsed from Text File}\label{fig:delim_text_file_selected2}\smallskip
   \includegraphics[scale=0.6]{qgis_user_guide_images/plugins_delimited_text/file_selected2}
   \end{center}  
\end{figure}

\begin{figure}[h]
   \begin{center}
   \caption{Selecting the X and Y Fields}\label{fig:delim_text_file_selected3}\smallskip
   \includegraphics[scale=0.6]{qgis_user_guide_images/plugins_delimited_text/file_selected3}
   \end{center}  
\end{figure}
\begin{figure}[h]
   \begin{center}
   \caption{Delimited Text Layer Added to QGIS}\label{fig:layer_added}\smallskip
\includegraphics[scale=0.6]{qgis_user_guide_images/plugins_delimited_text/layer_added}
   \end{center}  
\end{figure}



% html: End of file: `index.html'
